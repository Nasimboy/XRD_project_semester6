\section{Type of X-ray diffraction experiments}

Based on the type of crystal, X-ray diffraction experiments for crystallography can be divided into two categories:%
%	
	\begin{itemize}%
%	
	    \item \bfnt{Powder XRD (PXRD)}:%
%	    	
	    	\begin{itemize}[label={$\rightarrowtail$}]%
%	    	
	    	    \item We get information about bulk properties like phase, purity, particle size, polymorph detection, etc.
	    	    
	    	    \item A bulk sample composed of micron or sub-micron sized particles is used.
	    	    
	    	    \item Small angle PXRD: $0-3~\si{\degree}$ in $2\theta.$ Wide angle PXRD: $3-80~\si{\degree}$ in $2\theta.$
	    	    
	    	    \item Generally, $\mathrm{Cu}~K_\alpha$ radiation is used.
	    	    
	    	\end{itemize}
	    	
	    \item \bfnt{Single Crystal XRD (SCXRD)}:%
%	    	
	    	\begin{itemize}[label={$\rightarrowtail$}]%
%	    	
	    	    \item Complete information about the crystal structure can be obtained.
	    	    
	    	    \item Slightly bigger crystals of size $5-50~\si{\micro\metre}$ have to be used.
	    	    
	    	    \item Generally $\mathrm{Mo}~K_\alpha$ radiation is used.
	    	    
	    	    \item $2-50~\si{\degree}$ in $2\theta$ sufficient for routine structure analysis. High resolution SCXRD requires $2\theta \in \qty[ \SI{2}{\degree}, \SI{120}{\degree} ].$
	    	    
	    	\end{itemize}
	    
	\end{itemize}
	
There are two variables in Bragg's Law: $\theta$ and $\lambda,$ while $d_{hk\ell}$ is fixed. In order to satisfy Bragg's Law for any of the $d$-values, either $\theta$ has to be varied keeping $\lambda$ fixed, or vice-versa. Thus, X-ray diffraction experiments can be subdivided into a different set of two categories:

	\begin{itemize}
	
	    \item \ul{Fixed $\theta$, varying $\lambda$ techniques} -- the \bfnt{Laue method}
	    
	    Laue method utilizes white X-rays rather than monochromatic or near-monochromatic radiation. The important point to emphasize is that each set of reflecting planes with Laue indices $h k \ell$ gives rise to just one reflected beam. Of all the white X-radiation falling upon it, a lattice plane with Miller indices $(h k \ell)$ reflects only that wavelength for which Bragg’s law is satisfied, a reflecting plane of half the spacing with Laue indices $2h ~ 2k ~ 2l$ reflects a wavelength of half this value, and so on. In other words the reflections from planes such as, for example, $(111)$ (Miller indices for lattice planes) $222$, $333$, $444$, etc. (Laue indices for the parallel reflecting planes) are all superimposed.
	    
	    \textcolor{red}{The spots in the Laue method reveal only the angles between different $\qty{hkl}$ planes} (the measured angles being twice those between the planes themselves) and from which, and also from the symmetry relations between the spots, they may be indexed and the orientation of the crystal, w.r.t. the incident beam, determined. \textcolor{red}{The method does \textbf{not} allow us to determine $d_{hk\ell}$-spacings} as the wavelength(s) giving rise to each spot are unknown. In short, crystals with identical structures but different lattice parameters give identical Laue patterns.
	    
	    The method can be classified into two distinct groups: 1. the (original) transmission method, and 2. the back-reflection method. In both cases reflections from planes in the same zone lie on the surface of a circular cone centred about the zone axis. The intersection of the cone with the film (image plate) is a conic section. For small reflection angles (transmission), i.e. the planes in the zone making small angles to the incident beam, the conic section is an ellipse which passes through the origin (i.e. the direction of the incident beam). As the angle of the cone increases the ellipse `spreads out' as a parabola, then a hyperbola, and when the axis of the cone is at $\SI{90}{\degree}$ to the incident beam the spots fall in straight lines passing through the origin.
	    
	    Although it was the first, the transmission Laue method became little used (in comparison with ``fixed $\lambda$, varying $\theta$'' methods described in the following sections). However, with the advent of `wiggled', varying wavelength synchrotron radiation it has had something of a renaissance particularly for protein crystals since data can be acquired in a fraction of a second, i.e. before any degradation of the crystal can take place. 

	In comparison, the back-reflection technique has been widely used in metallurgy/ materials science to determine the orientations and orientation relationships between crystals, e.g. the orientations of crystals in turbine blades or the orientation relationships between parent and product crystals in phase transformations. Again, the spots from planes in the same zone lie in hyperbolae except that (unlike the transmission case) these do not intersect the centre of the film/image plate except for the special case where the zone axis is $\SI{90}{\degree}$ to the incident beam. In analysing the film it is necessary to determine the projection of the normal (or reciprocal lattice vector) of each of the reflecting planes on to the film from each reflection $S$ and then to plot these on a stereographic projection. By measuring the angles between the normals, and then comparing them with lists of angles, it is possible to identify the reflections. In practice such manual procedures have largely been replaced by computer programs which determine the orientation of the crystal using as input data the positions of spots on the film, film-specimen distance and (assumed) crystal structure. The manual procedures are extensively elaborated in \cite{Cullity2014}.
	
	\item \ul{Fixed $\lambda$, varying $\theta$ techniques}
	
	Our aim is to bring as many reciprocal lattice points on the surface of the Ewald sphere as possible. To do this, we rotate the crystal in different angles, so that the X-ray beam is incident throughout the crystal in different angles, thereby satisfying Bragg's Law for as many crystal planes as possible.
	
	This technique can be officially classified into three categories:%
%		
		\begin{enumerate}%
%		
		    \item \textit{Oscillation method}: The crystal is oscillated about a mean position by a certain degree. During the oscillation, different reciprocal lattice points come on the surface of Ewald's sphere, and momentarily a diffracted beam is generated, which is recorded by a cylindrical detector. As the size of real space unit cell grows, the reciprocal lattice points come closer and closer, and often overlap for complex biological molecules.
		    
		    \item \textit{Rotation method}: The crystal is slowly rotated by $2\pi$ about the axis it is mounted on the diffractometer, thereby giving us the rotation photograph of the crystal. This has been discussed under ``Selection of crystal" in SCXRD.
		    
		    \item \textit{Precession method}: The crystal and the film are rotated in a particular direction slowly so that all the reciprocal lattice points at a particular plane (or ``level'') fall on the circumference of Ewald's sphere.
		    
		\end{enumerate}
		
	In addition to these sub-categories, most SCXRD and PXRD diffractometers widely available today utilize the varying $\theta$ method. These diffractometers, described later, have multiple axes of rotation so that the whole crystal can be effectively brought under the X-ray beam and as many crystal planes are harvested as possible.
	    
	\end{itemize}