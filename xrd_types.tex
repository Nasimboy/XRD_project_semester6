\section{Type of X-ray diffraction experiments}

X-ray diffraction experiments for crystallography can be divided into two categories:%
%	
	\begin{itemize}%
%	
	    \item \bfnt{Powder XRD (PXRD)}:%
%	    	
	    	\begin{itemize}[label={$\rightarrowtail$}]%
%	    	
	    	    \item We get information about bulk properties like phase, purity, particle size, polymorph detection, etc.
	    	    
	    	    \item A bulk sample composed of micron or sub-micron sized particles is used.
	    	    
	    	    \item Small angle PXRD: $0-3~\si{\degree}$ in $2\theta.$ Wide angle PXRD: $3-80~\si{\degree}$ in $2\theta.$
	    	    
	    	    \item Generally, $\mathrm{Cu}~K_\alpha$ radiation is used.
	    	    
	    	\end{itemize}
	    	
	    \item \bfnt{Single Crystal XRD (SCXRD)}:%
%	    	
	    	\begin{itemize}[label={$\rightarrowtail$}]%
%	    	
	    	    \item Complete information about the crystal structure can be obtained.
	    	    
	    	    \item Slightly bigger crystals of size $5-50~\si{\micro\metre}$ have to be used.
	    	    
	    	    \item Generally $\mathrm{Mo}~K_\alpha$ radiation is used.
	    	    
	    	    \item $2-50~\si{\degree}$ in $2\theta$ sufficient for routine structure analysis. High resolution SCXRD requires $2\theta \in \qty[ \SI{2}{\degree}, \SI{120}{\degree} ].$
	    	    
	    	\end{itemize}
	    
	\end{itemize}