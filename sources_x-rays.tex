\section{Sources of X-rays}

	For X-ray diffraction experiments, monochromatic radiation from various metal-based sources (eg. Cu, Ag, Mo) is used in the lab. There are also synchotron-based sources for X-rays.%
%			
	\begin{itemize}%
%			
	    \item \bfnt{Sealed-tube X-ray sources}%
%			    	
	    	\begin{itemize}[label={$\hookrightarrow$}]%
%			    	
	    	    \item \ul{Fine focus sources}: Operate at around $50~\si{kV}$ and $40~\si{mA}$ ($\sim \SI{2}{kW}$). Can give a photon flux of around $\SI{1e7}{photons/s~mm^2}.$
	    	    
	    	    \item \ul{Microfocus sources}: Highly focussed beam. Operates at $40-50~\si{kV}$ and $2-15~\si{mA}$  ($<\SI{1}{kW}$). Can provide around $\SI{1e8}{photons/s~mm^2}.$
	    	    
	    	\end{itemize}
	    	
	    \item \bfnt{Rotating anode based source}: The anode is rotated at a very high speed of around $10,000~\si{rpm}.$ Cu, Mo or dual anode is used. Generally microfocus-based system. Operates at $\SI{60}{kV}$ and $\SI{100}{mA}$ or higher ($> \SI{6}{kW}$). Can produce a flux $\sim \num{1e10}-\num{e11}~\si{photons/s~mm^2}.$
	    
	    \item \bfnt{Metal Jet source}: Liquid Gallium is used, giving $\lambda = \SI{1.340}{\angstrom}.$ High intensity at much low power. Can generate $\sim \num{1e11}-\num{e12}~\si{photons/s~mm^2}.$
	    
	    \item \bfnt{Synchrotron Sources}: Synchotron radiation is an EM radiation emitted when charged particles are subjected to an acceleration perpendicular to their velocity. In synchrotrons, this radiation is generated in the dipole bending magnets, undulators and wigglers. This radiation has a characteristic polarization and the wavelengths generated can span the entire EM spectrum. X-rays from synchrotron sources are produce high quality data in powder X-ray diffraction experiments.
	    
	\end{itemize}