\subsection{How much data do we have to collect?}

Understanding how much data we have to collect for different crystal systems was extremely essential even, say, 20 years back. In those days, the detector in the diffractometer used to be a point detector. Such a detector can record only one reflection at a time, and it would take quite a long time to center a reflection at the maxima and then record the intensity. 

When we mount a crystal on the diffractometer and rotate it about the $\phi$ axis, diffraction would be visible from all directions. The detector, however, being a point detector, could only measure in a particular plane. The procedure used was as follows: First, the detector would be moved to the location of a particular reflection. Next, the detector would be moved in very small steps to locate the maxima of the diffracted light. Thereafter, the crystal would be very slowly rotated about the different axes of the diffractometer to maximize the intensity once again. Thus, for recording a particular reflection, it would take 5-7 minutes. If the intensity is weak at that point, it would take at least 15 minutes to get sufficient data. So, without Friedel's Law, one would have to spend more than a week to record all reflections.

Suppose, for a tricilinic system, 10,000 reflections are possible considering all $(h,k,\ell)$ and $(\bar{h}, \bar{k}, \bar{\ell}).$ It would take around 6-8 days for this data to be collected. Thanks to Friedel's Law, we know that we have to collect only half of the data rather than the full sphere. If we know whether the crystal is monoclinic, triclinic or orthorhombic, data collection time reduces further.

Even with the advent of modern area detectors, Friedel's Law is still equally useful.