\section{Spectroscopy vs. Crystallography}
	
\textbf{Spectroscopy} is the study of absorption and/or emission of electromagnetic radiation in which the incident radiation interacts with the molecule and produces characteristic signals. From spectroscopy, we can get the following data about a molecule:%
%			
\begin{itemize}%
%		
    \item Bond lengths and bond angles of simple (diatomic, triatomic) molecules.
    
    \item Combination of Infrared and NMR (${}^1 \mathrm{H}$ and ${}^{13} \mathrm{C}$) spectroscopy provides information about functional groups, bond connectivity and stereochemistry (absolute configuration) of simple molecules.
    
    \end{itemize}
    
Spectroscopy, however, cannot provide information about bond lengths, bond angles, torsion angles, etc. of complicated molecules.

\textbf{Crystallography}, on the other hand, deals in the interaction of EM radiation of very small wavelength ($\lambda \sim \SI{1}{\angstrom}$) with solid materials (crystalline or amorphous) through scattering. The main difference with spectroscopy is that in crystallography, there is no absorption ot emission of radiation; only scattering of radiation. Crystallography allows us to study%
%			
\begin{itemize}%
%			
    \item Accurate bond length, bond angles, torsion angles, precise absolute configuration, etc. of all types of crystalline substances.
    
    \item 3D structure of crystalline substances can lead to intermolecular or interionic interactions. Structure property relationship can be obtained using data from crystallography.
    
\end{itemize}

\section{Why X-rays?}

Simply because the the wavelength of X-rays is of the same order as the inter-atomic distance in a crystal lattices.