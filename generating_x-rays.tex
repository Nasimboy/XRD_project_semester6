\section{Generating X-rays}

	Fig.~XXXX shows the schematic of a X-ray tube. The potential difference between the anode to cathode is around $20-60~\si{kV},$ while the Tungsten filament is supplied a current of $\sim 2-50~\si{mA}.$ The target is composed of the material from which we want to generate the X-rays (generally Cu, Mo or Ag). The Be-window provides a transparent window for the generated X-rays to pass through.
	
	The tube is not allowed to cool down between experiments. In the stand-by state, the filament current is reduced to around $\SI{5}{mA}$ and the anode potential is also reduced to $\SI{20}{kV}.$ When data collection is started, the anode voltage is increased to $>\SI{50}{kV}$ and the current in the Tungsten filament is increased to $\SI{40}{mA}.$ In this state, the heated Tungsten filament generates electrons, which are then accelerated and finally hit the target at the anode.
	
	These electrons are able to knock out electrons from the K-shell of an atom in the target. Once an electron is removed from the K-shell, electrons from higher energy levels release energy and come to the K-shell. This release energy is the characteristic X-rays of the material. There are three possible transitions which give rise to X-rays of three different wavelengths. These are shown in Fig.~XXXX.
	
	The X-ray spectra of Cu and Mo are shown in Fig.~XXX. The $K_\alpha$ peaks are very close to each other, and are shown magnified in Fig.~XX. The background radiation is attributed to Bremstrahlung, which is the radiation emitted by electrons when they are decelerated in the X-ray tube. The corresponding wavelengths can be found in table~XX.
	
	We want to use monochromatic radiation for X-ray diffraction experiments. For this, we have to filter out the unwanted $K_\beta$ and $K_{\alpha2}$ radiations. To filter the $K_\beta$ spectra, we use a $\beta$-filter. The  $\beta$-filters used for Cu and Mo are listed in table~XX. To separate $K_{\alpha1}$ from $K_{\alpha2}$, we use a crystal monochromator.