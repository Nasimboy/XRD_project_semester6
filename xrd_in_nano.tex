\section{Use of XRD in characterization of nanomaterials}

Powder XRD is commonly used in characterization of nanomaterials. Analysis of a powder sample by PXRD provides important information like phase identification, sample purity, crystallite size, and sometimes, even the morphology.~\cite{Holder2019}.

The primary issue with using XRD in nanoscale systems is the peak broadening with the decrease in the crystallite size. As long as adjacent peaks are not overlapping, the Scherrer equation can be used to find the crystallite domain size. Below $\SI{10}{nm},$ however, peak broadening is so significant that signal intensity is low, and peaks often overlap. For domain sizes below $\SI{5}{nm},$ it is difficult to analyze the data, both because of broad peaks and low signal-to-noise ratio.

Size-dependent XRD peak broadening has important implications for nanomaterial characterization. For instance, if TEM analysis shows spherical particles having an average diameter of $\SI{10}{nm},$ but the XRD pattern has sharp peaks that are more consistent with particles having much larger crystalline domain sizes, then the majority of the bulk sample is not composed of $\SI{10}{nm},$ particles; it is more likely that the microscopically observed $\SI{10}{nm},$ particles represent only a minority subpopulation.

Not all nanoparticles are spherical. In case of non-spherical nanoparticles, there is a chance that upon drying, they will orient in non-random directions. As noted in~\cite{Holder2019}, a sample of cube-shaped particles dried or precipitated from solution will tend to orient with their flat faces parallel to the drying surface. It is much less likely that nanocubes would dry with their corners or edges touching the drying surface, and therefore, the powder of nanocubes will be preferentially oriented in the crystallographic direction corresponding to the faces. Similarly, one-dimensional nanowires will tend to orient flat on a substrate upon drying. Other particle shapes, such as octahedra or tetrahedra, may have different ways of orienting. The majority of the sample may exhibit preferred orientation, or only a fraction of it may, depending on the quality and size of the various particle shapes. In addition, the method in which the sample was dried to form a powder and/or how the XRD sample was prepared can influence the preferred orientation of the sample. We have already discussed the preferred orientation effect in PXRD. This can then be utilized to gain information about the structure of the crystallized nanoparticles.

Another application of PXRD is \bfnt{phase identification}, which is often accomplished by comparing an experimental XRD pattern with a reference pattern that is either simulated or obtained from a database. In such cases, an unambiguous and complete match between the experimental and reference patterns is needed. Arbitrary peaks predicted by a reference pattern cannot be missing in the experimental XRD data without justification. All peaks in the reference pattern, which includes both of their diffraction angles and intensities, should be accounted for in the experimental pattern unless there is a clear and justified rationale for why certain peaks may be missing or have different intensities, such as preferred orientation, as discussed above. To accomplish this comparison, experimental XRD patterns having sufficient signal-to-noise ratios are needed so that low-intensity peaks can be observed.

Phase identification by XRD for some systems, especially nanoscale materials, can be particularly challenging because of nearly indistinguishable diffraction patterns. For example, Au and Ag are both face-centered cubic metals that have sufficiently similar lattice constants that Au and Ag nanoparticles (which have broadened peaks) cannot be differentiated by XRD.

PXRD patterns for crystalline and amorphous samples are highly dissimilar. This can be used to detect the presence of amorphous impurities in the sample. However, samples that produce XRD patterns having low signal-to-noise ratios, including poorly crystalline materials and nanoscale materials having significantly broadened peaks, can contain large amounts of components that do not produce XRD peaks that rise significantly above the background noise. Low-intensity peaks, which may correspond to impurities, can also be difficult to observe. The presence of asymmetric peaks may be due to stacking faults and other defects or a distribution of compositions in compounds that could be present as alloys or solid solutions.

For bulk crystalline samples, lattice constants can be calculated upto four decimal places. However, for nanoscale materials, peak broadening and low signal-to-noise ratio does not allow the values to be so precise.